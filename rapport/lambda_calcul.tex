% !TEX encoding = UTF-8 Unicode
Cette partie a été faite par Wissame, l'idée du lambda calcul est apparue au milieu du projet quand il a fallut gérer les boutons du menu, le fait d'associer un bouton à quelque chose que l'on pouvait executer directement était séduisante pour la clarté du code mais aussi pour le défis technique que cela suscitait. Après quelques recherches, la classe Runnable est apparue comme nécessaire pour la réalisation.
\ml
La classe Runnable est une classe qui existe depuis des années dans le langage JAVA mais depuis JAVA 1.8 il est possible de définir un Runnable à partir d'un lambda calcul. 
\ml
Un Runnable peut donc être défini par un lambda calcul, cependant ce calcul ne peut prendre aucune entrée et ne renvoie rien en sortie. 

\subsubsection{Le lambda calcule en JAVA}

Depuis l'API JAVA 1.8, il est possible en JAVA de faire du lambda calcul, pour cela on doit passer par des objets spécifiques. Dans ce projet, nous avons eu besoin des objets Runnable, Callable et Consumer.

\begin{description}
\item[Runnable] Défini par un lambda calcul qui n'accepte aucun paramètre et qui ne renvoie rien.
\item[Callable] Défini par un lambda calcul qui ne prend rien et renvoie quelque chose.
\item[Consumer] Défini par un lambda calcul qui prend quelque chose en entrée mais qui ne renvoie rien.
\end{description}

Il existe bien évidement d'autres types d'objets pour être définis par des lambda expressions plus complexes.

Nous avons par exemple utilisé les Runnables pour définir l'execution de l'action adéquat en cas de collision avec un objet. Mais aussi pour gérer de manière plus générale la victoire et la mort du joueur.
\ml
Les Consumers ont servis pour le parcours de liste, avec la méthode foreach d'un objet Collection. Il est en effet possible d'effectuer une action pour chacun des objets de la liste sans avoir besoin d'un Iterator, en plus d'un gain de code, on limite le nombre d'erreurs en supprimant les effets de bords.
\ml
Les Callables quant à eux ont été nécessaires pour la parallèlisation des tâches.

\subsubsection{ExecutorService et Future}

Comme vue précédemment les Callables ont permis la parallèlisation des tâches. Associés à un ExecutorService on peut en effet parallèliser simplement des tâches définies par des lambda expressions.
\ml
Il existe plusieurs manière d'utiliser un ExecutorService, dans ce projet je l'ai utilisé de deux manières, avec un submit et avec un invokeAll.
\ml
La fonction submit d'un ExecutorService permet d'executer une action sur un autre thread, cette fonction renvoie un objet de type Future qui renseigne sur l'état d'avancement de la tâche, on peut donc savoir quand est-ce qu'elle est finie si on a besoin d'attendre la fin avant de passer à la suite.
\ml
De plus, l'ExecutorService permet d'executer une collection de Callable grace à la fonction invokeAll, cette fonction prend donc une Collection de Callable et va l'executer sur le nombre maximal de Thread sur lequel l'ExecutorService a été définie. On n'a pas besoin de récupérer de Future. Cette fonction est bloquante, on ne passe à la suite que si toutes les executions des Callables sont finies.
