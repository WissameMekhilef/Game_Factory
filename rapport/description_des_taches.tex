% !TEX encoding = UTF-8 Unicode

\subsubsection{Recherche d'un framework}
Nous avons choisi d'utiliser la librairie LWJGL \footnote{Lightweight Java Game Library en version 2.9.3} afin de gérer l'affichage graphique de notre jeu. Nous avons ensuite suivi le premier tiers du tutoriel "Jeu 2D avec LWJGL" sur la chaine YouTube "Tuto Programmation (Marccspro)" afin de nous familiariser avec cette librairie.

\subsubsection{Création d'un déplacement respectant les lois de la physique}
Cette partie a été réalisée par Wissame, dès que le joueur fut affiché à l'écran, il a voulu qu'il puisse se déplacer de manière naturelle. Pour cela, il a utilisé Matlab pour tester les fonctions de gravité avant de les implementer dans le jeu.

\subsubsection{Gestion d'un mouvement de caméra simple et autonome}
Dans l'optique de pouvoir évoluer dans un monde plus grand que la fenêtre du jeu, Simon a implémenté un scrolling horizontal à vitesse fixée.

\subsubsection{Gestion de l'interaction entre le joueur et la fenêtre}
À ce stade du projet, nous avons créé un dépôt Git afin de mettre en commun notre travail, l'objectif à court terme étant de gérer simultanément la physique du joueur et le scrolling de la fenêtre, ainsi que les collisions entre les bords de cette dernière et le joueur.

\subsubsection{Gestion des collisions}
Après avoir implémenté des obstacles, Wissame et Simon ont décidé de gérer les collisions entre le joueur et ces derniers. Cette tâche a été le premier défi technique que nous avons rencontré.

\subsubsection{Création d'un parser JSON}
Afin de pouvoir choisir entre plusieurs mondes au lancement du jeu, Simon a décidé de stocker les données des différents niveaux dans des fichiers JSON qui seraient lu au moment où l'utilisateur choisit le monde auquel il veut jouer.
\ml
Chaque fichier contient les toutes les informations nécessaires à l'instanciation du monde souhaité, telles que les textures à afficher, la musique à jouer, le placement des obstacles et de la sortie, le type de caméra, ou encore la puissance de la gravité qui s'applique au joueur.
\ml
Ce système nous a permis par la suite de créer des mondes variés et de s'affranchir des contraintes liées au fait de stocker les données des différents mondes directement dans le code du jeu.

\subsubsection{Ajout des textures, du son et des polices d'écriture}

Après que Vincent a essayé pendant plusieurs semaines d'implémenter des textures afin que le jeu puisse rendre à l'écran d'autres éléments que des tuiles monochromes, Wissame a décidé d'utiliser la librairie Slick afin de résoudre ce problème bloquant, la librairie Slick a permis de récupérer des objets texture qui ont pu directement être utilisé dans la fonction de rendue.
\ml
En plus de lui permettre de résoudre rapidement le problème, Slick nous a permis par la suite d'ajouter un fond sonore à notre jeu, ainsi que de rendre du texte avec une police particuliere, de la même manière que pour les textures des objets Font et Sound sont disponible grâce à cette API.
\ml
Slick étant concue comme une extension de LWJGL l'ajout toutes ces fonctions n'a pas posé de problèmes.