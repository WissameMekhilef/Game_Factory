% !TEX encoding = UTF-8 Unicode
\documentclass[french,12pt]{article}

\usepackage[utf8]{inputenc}
\usepackage{color}
\usepackage{fancyvrb} % pour mettre les verbatim dans des boites
\usepackage[pdftex]{graphicx}
\usepackage{listings}
\usepackage{float}
\usepackage{listingsutf8}
\usepackage[french]{babel}


\newcommand{\monimage}[4]{
\par\noindent
\begin{figure}[p] %on ouvre l'environnement figure
\begin{center}
\includegraphics[width=15cm]{#1} %ou image.png, .jpeg etc.
\caption{#2} %la légende
\label{#3} %l'étiquette pour faire référence à cette image
\end{center}
\end{figure} %on ferme l'environnement figure
}

\newcommand{\ml}[0]{\par\noindent}

\begin{document}

\thispagestyle{empty}
%
\begin{figure}[H]
\includegraphics[width=0.2\linewidth]{logo_univ.png}
\hfill
\includegraphics[width=1.5in]{logo_ufr.png}
\end{figure}
\vspace{2cm}
%
\begin{center}
{\Huge Rapport Final}
\par\vspace{0.4cm}
{\Large du}
\par\vspace{0.4cm}
{\Huge\bf Projet de L3}
\par\vspace{3cm}
{\Huge       GAME FACTORY}
\par\vspace{3cm}
{\Huge       Réalisé par:}

\par\vspace{0.3cm}
{\Huge\bf \textcolor{blue}{Simon ARNOULT}}
\par\vspace{0.3cm}
{\Huge\bf \textcolor{blue}{Wissame MEKHILEF}}
\par\vspace{0.3cm}
{\Huge\bf \textcolor{blue}{Vincent RENARD}}

\end{center}
\newpage
\tableofcontents
\newpage


\section{Introduction}

Sans avoir à rappeler le sujet, nous avons choisit de programmer un monde 2D avec une vue latérale, les mouvements de caméra au début n'étaient pas clairement définis.
\ml
Quant au gameplay, on voulait un jeu simple à jouer avec une idée empruntée au jeu de runner où seul le saut est possible.

\section{Présentation globale}
\subsection{Diagram classe}
% !TEX encoding = UTF-8 Unicode
Le diagramme de classe ci-dessous donne un aperçu de l'état des classes, de leur association mais aussi de qui créer chacun des objets.
\ml
On se rend compte que WorldReader cree tout ce qui est nécessaire pour le monde.

\monimage{uml_diagram.png}{Le diagramme de classe UML}{UML classe}{1}

\subsection{Diagramme des boucles de jeu}
% !TEX encoding = UTF-8 Unicode
\subsubsection{Les contextes}

Le contexte est un concept que nous avons utilisé pour déterminer l'état dans lequel se trouve le jeu. Il détermine quelles données doivent être mises à jour et quels éléments du monde doivent être affichés.
\ml
À chaque contexte est associé un objet principal, ces associations sont décrite sur la figure \ref{CONT1}.
\ml
Les transitions d'un contexte à un autre sont décrites sur la figure \ref{CONT2}.

\monimage{contexte_1.eps}{Schéma de l'association contexte et objet principal associé}{CONT1}{10}
\monimage{contexte_2.eps}{Schéma des transitions entre contexte}{CONT2}{11}

\subsubsection{La boucle update}

La vitesse du jeu dépend de la fréquence des « ticks ». Chaque tick est un marqueur temporel séparé de son prédécesseur par une très courte durée fixée – dans notre programme, deux ticks consécutifs sont séparés par un soixantième de seconde.
\ml\ml
La mise à jour des données du jeu s'effectue à chaque tick. Cela permet à notre programme de s'exécuter à une vitesse constante, et surtout indépendante de la puissance de la machine de l'utilisateur.
\ml\ml
La figure \ref{DSSUPDATE} montre comment se déroule la mise à jour des données en cours de partie.

\monimage{dss_update.eps}{Diagramme Séquence Système de la fonction Update}{DSSUPDATE}{12}

\subsubsection{La boucle render}

Contrairement à la mise à jour des données, l'actualisation de l'affichage n'a pas besoin d'être bridé : en effet, un nombre élevé de FPS (frames per second) permet d'obtenir un rendu à l'écran plus fluide.

\subsection{Architecture MVC}
% !TEX encoding = UTF-8 Unicode
\subsubsection{Affichage des données}

Afin de séparer la vue des données, nous avons décidé de déléguer les méthodes d'affichage à une classe non instanciable nommée « Graphics » (cf. figure \ref{DSSRENDER}).

\monimage{dss_render.eps}{Diagramme Séquence Système de la fonction Render}{DSSRENDER}{17}





\section{Une organisation sans faille}

\monimage{gantt.pdf}{Diagramme de GANTT}{dg}{1}

On a toujours durant le projet avancer par étapes nécessaire pour ne pas se perdre.

\input{diagram_gantt_explication.tex}

\subsection{Description des taches}

% !TEX encoding = UTF-8 Unicode

\subsubsection{Recherche d'un framework}
Nous avons choisi d'utiliser la librairie LWJGL \footnote{Lightweight Java Game Library en version 2.9.3} afin de gérer l'affichage graphique de notre jeu. Nous avons ensuite suivi le premier tiers du tutoriel "Jeu 2D avec LWJGL" sur la chaine YouTube "Tuto Programmation (Marccspro)" afin de nous familiariser avec cette librairie.

\subsubsection{Création d'un déplacement respectant les lois de la physique}
Cette partie a été réalisée par Wissame, dès que le joueur fut affiché à l'écran, il a voulu qu'il puisse se déplacer de manière naturelle. Pour cela, il a utilisé Matlab pour tester les fonctions de gravité avant de les implementer dans le jeu.

\subsubsection{Gestion d'un mouvement de caméra simple et autonome}
Dans l'optique de pouvoir évoluer dans un monde plus grand que la fenêtre du jeu, Simon a implémenté un scrolling horizontal à vitesse fixée.

\subsubsection{Gestion de l'interaction entre le joueur et la fenêtre}
À ce stade du projet, nous avons créé un dépôt Git afin de mettre en commun notre travail, l'objectif à court terme étant de gérer simultanément la physique du joueur et le scrolling de la fenêtre, ainsi que les collisions entre les bords de cette dernière et le joueur.

\subsubsection{Gestion des collisions}
Après avoir implémenté des obstacles, Wissame et Simon ont décidé de gérer les collisions entre le joueur et ces derniers. Cette tâche a été le premier défi technique que nous avons rencontré.

\subsubsection{Création d'un parser JSON}
Afin de pouvoir choisir entre plusieurs mondes au lancement du jeu, Simon a décidé de stocker les données des différents niveaux dans des fichiers JSON qui seraient lu au moment où l'utilisateur choisit le monde auquel il veut jouer.
\ml
Chaque fichier contient les toutes les informations nécessaires à l'instanciation du monde souhaité, telles que les textures à afficher, la musique à jouer, le placement des obstacles et de la sortie, le type de caméra, ou encore la puissance de la gravité qui s'applique au joueur.
\ml
Ce système nous a permis par la suite de créer des mondes variés et de s'affranchir des contraintes liées au fait de stocker les données des différents mondes directement dans le code du jeu.

\subsubsection{Ajout des textures, du son et des polices d'écriture}

Après que Vincent ait essayé pendant plusieurs semaines d'implémenter des textures afin que le jeu puisse rendre à l'écran d'autres éléments que des tuiles monochromes, Wissame a décidé d'utiliser la librairie Slick afin de résoudre ce problème bloquant, la librairie Slick a permis de récupérer des objets texture qui ont pu directement être utilisés dans la fonction de rendue.
\ml
En plus de lui permettre de résoudre rapidement le problème, Slick lui a permis par la suite d'ajouter un fond sonore au jeu, ainsi que de rendre du texte avec une police particulière, de la même manière que pour les textures, des objets Font et Sound sont disponibles grâce à cette API.
\ml
Slick étant conçue comme une extension de LWJGL l'ajout toutes ces fonctions n'a pas posé de problèmes.

\section{Quelques défis techniques}
\input{quelques_defis_techniques.tex}

\subsection{Gestion des collisions}
% !TEX encoding = UTF-8 Unicode

Pour résoudre ce problème, Wissame et Simon ont implémenté des objets de type PotentialCollision.
Ce sont des couples qui associe au joueur un obstacle du monde. À chaque update du jeu, chaque PotentialCollision est interrogé afin de savoir si le joueur est en contact avec un obstacle, et le cas échéant, sur quel bord de ce dernier la collision a eu lieu.
\ml
De plus cette séparation des traitements obstacle par obstacle permet des les traiter dans un ordre quelconque, en effet si un PotentialCollision détecte une collision sur un coté de l'obstacle le Player reçoit cette information donc toutes les autres PotentialCollision reçoivent l'information. 
\ml
Donc si un PotentialCollision détecte une collision sur le haut de l'obstacle alors le Player reçoit l'information comme quoi il est bloqué par le bas et comme le Player reste un objet unique qui n'est pas dupliqué, toutes les autres PotentialCollision prennent en compte cette information.
\ml
C'est grâce à cela que le calcul des collision a put être parallélisé, la variable isStuckRoutine contient l'appel à toute les calculs de collision. Nous expliquons dans la partie qui suit comment cela fonctionne.

\subsection{Le lambda calcul, multithreadé}
% !TEX encoding = UTF-8 Unicode
Cette partie à était faite par Wissame, l'idée du lambda calcul est apparut au milieu du projet quand il a fallut gérer les bouton du menu, le fait d'associer un bouton à quelque chose que l'ont pouvait
executer directement était séduisante pour la clarté du code mais aussi pour le défis technique que cela représentais. Après quelque recherche la classe Runnable est apparût comme nécessaire pour
la réalisation.
\ml
La classe Runnable est une classe qui existe depuis des années dans le langage JAVA mais depuis JAVA 1.8 il est possible de définir un Runnable à partir d'une lambda expression. 
\ml
Un Runnable peut donc être définie par une lambda expression, cependant cette expression ne peut prendre aucune entrée et ne renvoie rien en sortie. Elle est donc adaptée si c'est une suite de fonction 
qui peut être exécutés indépendamment des autres. Rapidement il a fallut trouver d'autres types d'objet pour faire définir un objet q


\section{Conclusion}
Ce projet est maintenant fini et comme pour tout projet il y'a des objectifs qui ont été atteints et d'autres non.
\ml
Pour ce qui est des objectifs fixés, nous avons un jeu qui fonctionne. Le code est généralisé ce qui permettrait d'ajouter des nouveaux types d'objets facilement et de décrire l'interaction avec ces derniers. Les différents types de caméra sont eux aussi pleinement fonctionnels ce qui permet de proposer des gameplay différents. Les règles de la physique du monde sont elles aussi modifiables ce qui permet d'améliorer l'expérience de jeu. Pour finir le calcul parallèle permet de profiter des architectures multi-coeur. 
\ml
En ce qui concerne l'ajout de nouvelles fonctions, car il y'en aura toujours. On pourrait améliorer la gestion des collisions, car il y'a encore un bug avec deux obstacles contigues. On aurait pu jouer aussi sur le bruitage avec l'ajout de son lors d'une collision. On pourrait ajouter aussi de nouveaux types d'objets dans le monde tel que des monstres ou des pièces, tout le système de gestion étant déjà présent.

\newpage
\listoffigures
\newpage

\end{document}