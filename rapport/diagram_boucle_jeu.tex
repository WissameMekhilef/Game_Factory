% !TEX encoding = UTF-8 Unicode
\subsubsection{Les contextes}

Les contextes sont un concept que l'on a utilisé pour déterminer les états et un ainsi clairement séparé les moments dans le jeu pour ne faire les update et les rendue que des élements nécessaire.
\ml
De plus il nous ont permis de définir des règles pour passer d'un contexte a un autre.

\monimage{contexte_1.pdf}{schema 1}{CONT1}{10}
\monimage{contexte_2.pdf}{schema 2}{CONT2}{11}

\subsubsection{La boucle update}

La vitesse du jeu dépend de la fréquence des « ticks ». Chaque tick est un marqueur temporel séparé de son prédécesseur par une très courte durée fixée – dans notre programme, deux ticks consécutifs sont séparés par un soixantième de seconde.
\ml\ml
La mise à jour des données du jeu s'effectue à chaque tick. Cela permet à notre programme de s'exécuter à une vitesse constante, et surtout indépendante de la puissance de la machine de l'utilisateur.

\subsubsection{La boucle render}

Contrairement à la mise à jour des données, l'actualisation de l'affichage n'a pas besoin d'être bridé : en effet, un nombre élevé de FPS (frames per second) permet d'obtenir un rendu à l'écran plus fluide.