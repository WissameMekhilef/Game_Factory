% !TEX encoding = UTF-8 Unicode
\subsubsection{Les contextes}

Le contexte est un concept que nous avons utilisé pour déterminer l'état dans lequel se trouve le jeu. Il détermine quelles données doivent être mises à jour et quels éléments du monde doivent être affichés.
\ml
Les transitions d'un contexte à un autre sont décrites en figures \ref{CONT1} et \ref{CONT2}.

\monimage{contexte_1.pdf}{schema 1}{CONT1}{10}
\monimage{contexte_2.pdf}{schema 2}{CONT2}{11}

\subsubsection{La boucle update}

La vitesse du jeu dépend de la fréquence des « ticks ». Chaque tick est un marqueur temporel séparé de son prédécesseur par une très courte durée fixée – dans notre programme, deux ticks consécutifs sont séparés par un soixantième de seconde.
\ml\ml
La mise à jour des données du jeu s'effectue à chaque tick. Cela permet à notre programme de s'exécuter à une vitesse constante, et surtout indépendante de la puissance de la machine de l'utilisateur.
\ml\ml
La figure \ref{DSSUPDATE} montre comment se déroule la mise à jour des données en cours de partie.

\monimage{dss_update.eps}{Diagramme Séquence Système de la fonction Update}{DSSUPDATE}{12}

\subsubsection{La boucle render}

Contrairement à la mise à jour des données, l'actualisation de l'affichage n'a pas besoin d'être bridé : en effet, un nombre élevé de FPS (frames per second) permet d'obtenir un rendu à l'écran plus fluide.